\documentclass[11pt,a4paper]{article}
\input{../config}
\usepackage{amsmath}
\usepackage{moreverb}
\usepackage{float}


\renewcommand\subject[1]{MAT}

\begin{document} \sloppy
\titlepageandcontents

\section{Jazyk a sémantika predikátové logiky (termy, formule, realizace jazyka, pravdivost formulí)}
%=================================================
\section{Formální systém predikátové logiky (axiomy a odvozovací pravidla, dokazatelnost, model a důsledek teorie, věty o úplnosti a kompaktnosti, prenexní tvar formulí)}
%=================================================
\section{Algebraické struktury (grupy, okruhy, obory integrity a tělesa, svazy a Boolovy algebry, univerzální algebry)}
%=================================================
\section{Základní algebraické metody (podalgebry, homomorfismy, přímé součiny, kongruence a faktorové algebry, normální podgrupy a ideály okruhů)}
%=================================================
\section{Obory integrity a dělitelnost (okruhy polynomů, pravidla dělitelnosti, Gaussovy a Eukleidovy okruhy)}
%=================================================
\section{Teorie polí (minimální pole, rozšíření pole, konečná pole a jejich konstrukce)}
%=================================================
\section{Metrické prostory (příklady, konvergence posloupností, spojitá a izometrická zobrazení, úplnost, Banachova věta o pevném bodu)}
%=================================================
\section{Normované a unitární prostory (základní vlastnosti a příklady, normované prostory konečné dimenze, uzavřené ortonormální systémy a Fourierovy řady)}
%=================================================
\section{Obyčejné grafy (stupně uzlů, cesty a kružnice, souvislost grafu, stromy, kostry, Kruskalův a Primův algoritmus pro hledání minimální kostry ohodnoceného grafu, eulerovské a hamiltonovské grafy, obarvitelnost a planarita)}

\textbf{Obecný graf} je uspořádaná trojice $G = (V, E, f)$, kde
\begin{itemize}
\item $V$ je neprázdná konečná množina vrcholů (uzlů),
\item $E$ je konečná množina hran,
\item $f$ je \textit{incidenční} zobrazení $f$: $E \rightarrow V^2$, které přiřazuje každé hraně $e \in E$ uspořádanou dvojici vrcholů $(x,y) \in V^2$.
\end{itemize}

Uzel $x$ se nazývá \textbf{počáteční uzel hrany} $e$, uzel $y$ se nazývá \textbf{koncový uzel hrany} $e$.

O uzlech spojených hranou, říkáme, že spolu \textbf{sousedí}, jsou to tedy \textbf{sousední uzly}. Ve vztahu uzel - hrana se používá pojem \textbf{incidence}. Uzel, který není incidentní s žádnou hranou, je označován jako \textbf{izolovaný uzel}.

Je-li $x = y$, pak hrana $(x, y)$ je označována jako smyčka.

Definice incidenčního zobrazení připouští existenci dvou různých hran $h_1$ a $h_2$ takových, že $f(h_1) = f(h_2) = (x,y)$. V tom případě hovoříme o tzv. \textbf{násobných hranách}.

Graf bez násobných hran se nazývá \textbf{prostý}. Graf, který není prostý, se označuje jako \textbf{multigraf}. 

Graf je \textbf{úplný}, pokud mezi každými dvěma uzly existuje právě jedna hrana.

\textbf{Neorientovný graf} je graf $G = (V, E)$, který obsahuje neorientované hrany, tedy kde $E \subseteq {V \choose 2}$ a kde pro každou hranu $(x, y) \in E$ platí, že i $(y, x) \in E$. Neorientovaný graf neobsahuje smyšky.

\textbf{Orientovný graf} je graf $G = (V, E)$, který obsahuje orientované hrany, tedy kde $E \subseteq V \times V$.

\textbf{Sled}

\textbf{Tah}

\textbf{Cesta}

\textbf{Kružnice}

\textbf{Souvislý graf}

\textbf{Podgraf grafu}

\textbf{Faktor grafu} Podgraf indukovaný v''

\textbf{Komponenta grafu}

\textbf{Most}

\textbf{Stupeň vrcholů}

\textbf{Pro obyčejné grafy platí}
%=================================================
\section{Orientované grafy (orientované cesty a kružnice, souvislost a silná souvislost, turnaj, eulerovské a hamiltonovské grafy, Dijkstrův a Floyd-Warshallův algoritmus pro hledání cesty minimální délky)}

\textbf{Orientovaný graf} je graf $G = (V, E)$, kde $E \subseteq V^2$. Hrany $(u,u)$ se nazývají \textit{smyčky}.

\end{document}

