\documentclass[11pt,a4paper]{article}
\input{../config}
\usepackage{amsmath}
\usepackage{amsfonts}
\usepackage{moreverb}
\usepackage{float}


\renewcommand\subject[1]{MAT}

\begin{document} \sloppy
\titlepageandcontents

\section{Jazyk a sémantika predikátové logiky (termy, formule, realizace jazyka, pravdivost formulí)}
%=================================================
\section{Formální systém predikátové logiky (axiomy a odvozovací pravidla, dokazatelnost, model a důsledek teorie, věty o úplnosti a kompaktnosti, prenexní tvar formulí)}
%=================================================
\section{Algebraické struktury (grupy, okruhy, obory integrity a tělesa, svazy a Boolovy algebry, univerzální algebry)}
%=================================================
\section{Základní algebraické metody (podalgebry, homomorfismy, přímé součiny, kongruence a faktorové algebry, normální podgrupy a ideály okruhů)}
%=================================================
\section{Obory integrity a dělitelnost (okruhy polynomů, pravidla dělitelnosti, Gaussovy a Eukleidovy okruhy)}
%=================================================
\section{Teorie polí (minimální pole, rozšíření pole, konečná pole a jejich konstrukce)}
%=================================================
\section{Metrické prostory (příklady, konvergence posloupností, spojitá a izometrická zobrazení, úplnost, Banachova věta o pevném bodu)}
%=================================================
\section{Normované a unitární prostory (základní vlastnosti a příklady, normované prostory konečné dimenze, uzavřené ortonormální systémy a Fourierovy řady)}
%=================================================
\section{Obyčejné grafy (stupně uzlů, cesty a kružnice, souvislost grafu, stromy, kostry, Kruskalův a Primův algoritmus pro hledání minimální kostry ohodnoceného grafu, eulerovské a hamiltonovské grafy, obarvitelnost a planarita)}

\textbf{Obecný graf} je uspořádaná trojice $G = (V, E, f)$, kde
\begin{itemize}
\item $V$ je neprázdná konečná množina vrcholů (uzlů),
\item $E$ je konečná množina hran,
\item $f$ je \textit{incidenční} zobrazení $f$: $E \rightarrow V^2$, které přiřazuje každé hraně $e \in E$ uspořádanou dvojici vrcholů $(x,y) \in V^2$.
\end{itemize}

Uzel $x$ se nazývá \textbf{počáteční uzel hrany} $e$, uzel $y$ se nazývá \textbf{koncový uzel hrany} $e$.

O uzlech spojených hranou, říkáme, že spolu \textbf{sousedí}, jsou to tedy \textbf{sousední uzly}. Ve vztahu uzel - hrana se používá pojem \textbf{incidence}. Uzel, který není incidentní s žádnou hranou, je označován jako \textbf{izolovaný uzel}.

Je-li $x = y$, pak hrana $(x, y)$ je označována jako smyčka.

Definice incidenčního zobrazení připouští existenci dvou různých hran $h_1$ a $h_2$ takových, že $f(h_1) = f(h_2) = (x,y)$. V tom případě hovoříme o tzv. \textbf{násobných hranách}.

Graf bez násobných hran se nazývá \textbf{prostý}. Graf, který není prostý, se označuje jako \textbf{multigraf}. 

Graf je \textbf{úplný}, pokud mezi každými dvěma uzly existuje právě jedna hrana.

\textbf{Neorientovný graf} je graf $G = (V, E)$, který obsahuje neorientované hrany, tedy kde $E \subseteq {V \choose 2}$ a kde pro každou hranu $(x, y) \in E$ platí, že i $(y, x) \in E$. Neorientovaný graf neobsahuje smyšky.

\textbf{Orientovný graf} je graf $G = (V, E)$, který obsahuje orientované hrany, tedy kde $E \subseteq V \times V$.

Posloupnost uzlů v, kde $(v_{i-1}, v_i) \in E$ pro $i = 1,2,\dots,k$, se nazývá \textbf{sled} délky $k$ z \ $v_0$ do $v_k$. Pokud existuje sled $s$ z $u$ do $u'$, říkáme, že $u'$ je \textbf{dosažitelný} z $u$ sledem $s$, značeno $u \overset{s}{\leadsto} u'$.

\textbf{Tah} je sled, ve kterém se neopakují hrany.

\textbf{Cesta} je sled, ve kterém se neopakují uzly.

Sled (tah, cesta) $\langle v_0, v_1, v_2, \dots, v_k\rangle$ se nazývá \textbf{uzavřený}, pokud obsahuje alespoň jednu hranu a $v_0 = v_k$ (pro neorientovaný graf navíc požadujeme $k \geq 3$).

\textbf{Kružnice} je uzavřená cesta. Orientovaná kružnice se nazývá \textbf{cyklus}. Graf bez cyklů se nazývá \textbf{acyklický}.

Neorientovaný graf se nazývá \textbf{souvislý}, pokud mezi libovolnými dvěma uzly existuje cesta.

\textbf{Souvislé komponenty} grafu jsou třídy ekvivalence množiny uzlů podle relace "\textbf{je dosažitelný z}".

Orientovaný graf se nazývá \textbf{silně souvislý}, pokud mezi libovolnými dvěma uzly existuje orientovaná cesta.

\textbf{Silně souvislé komponenty} grafu jsou třídy ekvivalence množiny uzlů podle relace "\textbf{jsou vzájemně dosažitelné}".

\textbf{Podgraf grafu} $G = (V, E)$ je graf $G' = (V', E')$, kde $V' \subseteq V$ a $E' \subseteq E$.

\textbf{Faktor grafu} je podgraf $G' = (V, E')$ a je indukovaný množinou vrcholů $V$.

\textbf{Most} je hrana, jejíž odebráním zvýšíme počet komponent grafu.

\textbf{Stupeň vrcholu} $u$ je číslo $deg(u)$ definované jako počet hran incidentních s vrcholem $u$.

\textbf{Pro obyčejné grafy platí}: $\sum_{v \in V} deg(u) = 2m$, kde $m = |E|$.
%=================================================
\section{Orientované grafy (orientované cesty a kružnice, souvislost a silná souvislost, turnaj, eulerovské a hamiltonovské grafy, Dijkstrův a Floyd-Warshallův algoritmus pro hledání cesty minimální délky)}


\end{document}

